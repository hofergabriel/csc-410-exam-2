
\documentclass[12pt]{article}
\usepackage{listings}
\usepackage{pdfpages}
\usepackage{amsmath}
\usepackage[utf8]{inputenc}
\usepackage[english]{babel}
\usepackage{multicol}
\usepackage{babel}
\usepackage{graphicx}

\usepackage{tgschola}
%%%\usepackage{mathptmx}

\usepackage[margin=1in]{geometry}
\begin{document}
\begin{titlepage}
   \begin{center}
       \vspace*{1cm}
       \Large
       Exam 2: Applying Foster's Methodology 
       \normalsize

       \vspace{0.5cm}

       Author: Gabriel Hofer

       \vspace{0.5cm}

       Course: CSC-410 Parallel Programming

       \vspace{0.5cm}

       Instructor: Dr. Karlsson

       \vspace{0.5cm}

       Due: November 23, 2020

       \vfill

       Computer Science and Engineering\

       South Dakota School of Mines and Technology\
   \end{center}
\end{titlepage}
\newpage
%------------------------------------------------------------------------------------
%------------------------------------------------------------------------------------
\small
\newpage

\subsection*{Buidling the Histogram}
\subsubsection*{Partitioning}
We will partition the array into $n/chunksize$ subsections where $n$ is the size of the 
array of floats and the parameter $chunksize$ is the number of floats per subsection 
of the original array. If $n$ is not evenly divisible by $chunksize$, the remaining 
subsection will have fewer elements. 

Each subsection of the array is assigned a task. And, each task corresponds to a 
unique bin in the histogram. Task i is assigned to bin j such that i=j 
(i.e. tasks and bins have the same rank/identifier). See figure 1.


Before processing the data, each task has a local variable, cnt, which is initialized 
to zero. cnt will store the number of floating-point numbers in the bin that is 
associated with its task. 

However, the floating-point numbers in a subsection don’t necessarily ‘belong’ in 
the bin associated with their task number. We need communication in order to ‘move’ 
the floating-point numbers to the right tasks/bins. 

\subsubsection*{Communication}
Each task will iterate through its own subsection. For any floating-point number f 
in the subsection, the task calculates which bin it belongs to. The correct bin is 
calculated by dividing the floating-point number by the $bin\_width$ using integer 
division: $target\_bin = f / bin\_width$. After traversing through the whole sub-array, 
the task then sends a message to all bins. 
The messages contain the amount that each non-local task should update their local 
count (‘cnt’) variable. See figure 2. 

\subsubsection*{Agglomeration}

We could have created a task/process for each floating-point number in the array. 
This is actually possible if the $bin\_width$ is set to 1. However, it’s preferable to 
choose a larger value for $bin\_width$ because it will reduce the number of messages 
sent between processes. Notice that there will be $O(chunksize^2)$ messages sent 
between tasks for updating each tasks cnt. Also, the point of the histogram is to 
combine data points to show groups of similar data. A histogram with so many bins is 
less likely to be useful.

\subsubsection*{Mapping }

Since the number of tasks and the number of processors are the same, we simply assign 
task t to a process p such that t=p. This mapping was illustrated in Figure 1.

\subsubsection*{Combining Data in the Root Process}
The root task will construct a plot from the cnt of each task. For this to happen, 
the data needs to be retrieved and sent to the root. Once, each task has iterated through 
its own array, they will ping the root message, indicating that the task has processed all 
of its floating point numbers and that all messages to other processors have been received. 
Once the root process has been pinged this message from all of the processors/tasks, 
then the root will send back a pong to all of the tasks. 
The purpose of this pong is to tell every task to send their ‘cnt’ variable to the root task. 
The reason we need to do a ping-pong before each task simply sends its cnt variable to the 
root is that we need to make sure that every message has been completed 
(i.e. synchronized) before we send the cnt to the root. 










\end{document}





